\chapter{Dynamic Location Generation Implementation}
\label{appendix:location-generation}

I implemented a system that can either use predefined location data or generate it dynamically. This implementation allows for reproducible experiments with configurable topologies:

\begin{verbatim}
private static void generateLocationFiles(int numberOfEdge, int numberOfUsers, long seed) throws IOException {
    // First initialize the CoordinateConverter from the existing config file
    boolean initialized = CoordinateConverter.initializeFromConfig(LOCATION_CONFIG_FILE);
    
    // Define output directory for generated files
    String outputDir = OUTPUT_DIRECTORY;
    
    // Create the directory if it doesn't exist
    java.io.File dir = new java.io.File(outputDir);
    if (!dir.exists()) {
        if (dir.mkdirs()) {
            System.out.println("Created directory: " + outputDir);
        } else {
            System.err.println("Failed to create directory: " + outputDir);
            // Fall back to temporary files if directory creation fails
            outputDir = "";
        }
    }
    
    // Generate location configuration file
    DYNAMIC_LOCATION_CONFIG_FILE = CoordinateConverter.generateLocationConfig(
        outputDir + "location_config_" + seed + ".json");
    
    // Generate resource locations in a grid pattern
    DYNAMIC_RESOURCES_LOCATION_PATH = CoordinateConverter.generateResourceLocationsCSV(
        numberOfEdge, outputDir + "resources_" + seed + ".csv", seed);
    
    // Generate user locations with random distribution
    DYNAMIC_USERS_LOCATION_PATH = CoordinateConverter.generateUserLocationsCSV(
        numberOfUsers, outputDir + "users_" + seed + ".csv", seed);
    
    System.out.println("Generated dynamic location files in " + outputDir + " with seed: " + seed);
}
\end{verbatim}

This dynamic generation of location data allows experiments to be easily scaled to different numbers of edge servers and users, while maintaining reproducibility through the use of controlled random seeds. 